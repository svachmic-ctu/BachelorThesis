\chapter{Úvod}
Během posledních tří let výrazně stoupl počet chytrých mobilních telefonů (tzv. smartphonů) a tabletů mezi běžnými uživateli. Mobilní operátoři tento nárůst vnímají velice citlivě, protože smartphony a tablety mají vysokou datovou zátěž. Pro plnou funkčnost těchto zařízení je potřeba připojení k internetu, které lze realizovat přes WiFi nebo právě přes mobilní datovou síť operátora, která je plně saturována nahráváním dat na servery služeb typu Instagram, Facebook či Twitter. Operátoři pro zachování dobrého jména poskytují na informačních plakátech zavádějící parametry vlastní mobilní datové sítě. Tyto parametry jsou měřeny obvykle pomocí speciálního auta s BTS všesměrovou anténou na střeše. Je zřejmé, že tato anténa má kvalitnější příjem, než telefony padnoucí do uživatelovy kapsy. Měření je prováděno zpravidla v ideálních podmínkách - při slunečném počasí a mimo špičku. Díky těmto podmínkám dostane operátor nejlepší možná data (best-case scenario), které používá pro marketingové účely.

Projekt MBTest si dává za cíl objektivně změřit kvalitu připojení k internetu od mobilních operátorů a posoudit, zda jsou propagovaná data o mobilním internetu shodná, či protichůdná \emph{ve všech podmínkách}. Projekt MBTest se skládá z více částí a tato BP\footnote{Bakalářská práce} se zabývá konkrétní implementací tlustého klienta pro zařízení iPhone a iPad s operačním systémem iOS od společnosti Apple. Další části projektu jsou součásti BP ostatních členů týmu pracujícího na projektu MBTest. Mimo klienta pro iOS je ve vývoji nativní klient pro Android a statistický klient pro webový prohlížeč. Středobodem celého projektu je serverový backend, který implementuje David Watzke.

Chytré telefony a tablety poskytují vhodné prostředí pro vývoj aplikací třetích stran. Výrobci zařízení či jejich operačních systémů nabízí vývojářům nejenom nástroje pro vývoj(IDE, SDK), ale zárověň i přístup k výbavě telefonu. Tyto benefity dávají projektu MBTest prostor pro realizaci.

Zpracování měření na mobilních telefonech nedává totožnou zpětnou vazbu jako profesionální měřící nástroje, ale naopak vyjádří, jak přesně se dané zařízení chová. Mezi základní techniky měření bychom mohli zařadit stáhnutí souboru přes TCP, komunikaci přes UDP \\a změření odezvy při zavolání URL.